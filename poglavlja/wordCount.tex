\chapter{Решење проблема применом методе бројања речи}

Метода бројања речи једна је од једноставнијих метода којом  је могуће поредити два текстулана документа.  Основна претпоствка ове методе је да су документи ближи један другом уколико  имају више заједничких речи.  Оваква претпоствка иако делује сасвим основано, не мора увек да буде тачна. Истим скупом речи могу се описати потпуно различите ствари и тиме генерисати два текстулана документа која, по смислу, уопште нису слична. Иако су овакви примери бројни, ова метода је, пре свега, због једноставности имплементације широко прихваћена у системима за проналажења одговора на постављено питање.

У конкретном раду, метода бројања речи је коришћена као \textbf{компаративнно решење} у односу на решење применом алгоритма моделовања тема. 

\section{Опис решења методом бројања речи}

Улаз у алгоритам су група питања и група одговора, при чему се за свако питање унапред зна који одговор из дата групе одговора представаља тачан одговор. Решење методом бројања речи заснива се на мерењу сличности датог  питања са \textbf{сваким одговором} у бази одговора. Након тога, одговори се рангирају према израчунатој сличности. Позиција тачног одговра у тој хијерархији свих одговора предствља излаз који алгоритам даје за свако постављено питање. 

  