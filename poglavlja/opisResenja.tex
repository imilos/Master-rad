\chapter{Опис решења}
\subsection{Припрема подарака - предпроцесирање}

Подаци који су кориштени за тестирање приликом израде овог рада су јавно доступни подаци са сајта \textbf{\textit{http://stackexchange.com/}} из три области  - инжињерство, фитнес и хемија. Подаци су на \textbf{енглеском језику}.

Ове три области су намерно тако одабране како би се повећала разноврсност речи. Речи из сродних научних грана користе слично или исту терминологују па опасност од бирања питања и одговора из  сродних научних дисциплина лежи у чињеници да ће диверзитет корпуса бити мали.

Из сваке области узето је по 200 питаља и одговора, што представља базу од укупно 600 питања и 600 одговора. Овакав скуп података је затим подељен на три дела - тренинг део (360 питања и одговора) и тест део (240 питања и одговора).

Подаци добијени на овај начин нису погодни за директну обраду те их је потребно делимично \textbf{прерадити} или \textbf{предпроцесирати}. 

	\subsubsection{Избациваље често коришћених речи енг. stop words}
		
У свакодневном говору често се употребљавају личне заменице, прилози, везници итд. Без њих, говор би био неодређен и неповезан па самим и неразумљивив. Међутим, у машинској обради текстуланих података, оваква врста информација није неопходна. Напротив, уноси додатну забуну при закључивању и може представљати велико оптерећење приликом обраде.	
	
	
	\subsubsection{Додавање синонима}
У циљу бољег дифренцирања тематике питања и одговора, за сваку реч је додато по 5 њених синонима. За проналажење синонима је коришћена WordNet библиотека. Основни разлог додавања синонима у скуп била је претпоставка да ће се на тај начин боље диференцирати теме, повећати диверзитет корпуса а самим тим и олакшати препознавање тачног одговора. Међутим, резулатати су показали управо супротно. Разлог томе што синониме треба тражити \textbf{по смислу} речи а не само по лексичком облику речи. Такође, фразе, којих има доста у свакодневном говорном ишписаном енглеском језику, значајно могу да утичу на смисао питања/одговора. Када се они рашчлане на појединачне речи, могуће је да се и смисао промени.
	\subsubsection{Склањање наставака речи - енг. stemming}
	
У енглеском језику, различити облици речи граде се додавањем \textbf{наставака}. 

	
	\subsubsection{Свођење на коренску реч - енг. lemmitization}

