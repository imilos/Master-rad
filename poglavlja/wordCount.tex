\chapter{Решење проблема применом методе бројања речи}

Метода бројања речи једна је од једноставнијих метода којом  је могуће поредити два текстулана документа.  Основна претпоствка ове методе је да су документи ближи један другом уколико  имају више заједничких речи.  Оваква претпоствка иако делује сасвим основано, не мора увек да буде тачна. Истим скупом речи могу се описати потпуно различите ствари и тиме генерисати два текстулана документа која, по смислу, уопште нису слична. Иако су овакви примери бројни, ова метода је, пре свега, због једноставности имплементације широко прихваћена у системима за проналажења одговора на постављено питање.

У конкретном раду, метода бројања речи је коришћена као \textbf{компаративнно решење} у односу на решење применом алгоритма моделовања тема. 

\section{Опис решења методом бројања речи}

Улаз у алгоритам су група питања и група одговора, при чему се за свако питање унапред зна који одговор из дата групе одговора представаља тачан одговор. Решење методом бројања речи заснива се на мерењу сличности датог  питања са \textbf{сваким одговором} у бази одговора. Након тога, одговори се рангирају према израчунатој сличности. Позиција тачног одговра у тој хијерархији свих одговора предствља излаз који алгоритам даје за свако постављено питање. 

Да би резултати ове методе могли да се пореде са резултатима претходно развијеног решења, неопходно је обезбедити \textbf{исте улазне податке}. Обизором на начин предпроцесирања у решењу базираном на моделовању тема, да би се обезбедили идентични улази коришћен је исти приступ предпроцесирању. То подразумева развој класе која, након предпроцесирања података за коришћење у алгоритму моделовања тема, те податке уписује на екстерни диск. Овим је обезбеђен апсолутно исти улаз и за компаративно решење.

Ради лакшег рачунања сличности докумената. сваки документ је представљен као вектор, тј. као низ неких нумеричкоих вредности. Трансформације текстуалног документа у вектор може се обавити на више начина. У раду су кориштена два приступа :
\begin{itemize}
\item једноставно бројање речи - Сваки документ представљен је као један низ. Свакој речи документа се додељује један природан број који представља индекс у том низу, при чеми исте речи имају додељене исте бројеве. На тој позицији у низу налази се број појављивања те речи у документу. За мерење сличности два документа неопходно је обезбедити исто мапирање речи у природне бројеве. Ово значи да исти индекс у оба документа одговора истим речима.
\item коришћење популарне TF-IDF методе - вектори се формирају на исти начин као код класичног приступа бројањем речи стим што се нумеричка вредност у вектору рачуна по формулама TF-IDF методе.
\end{itemize}

За меру сличности два документа узета је \textbf{косинусна сличност} прерачунатих вектора.
